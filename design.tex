Basic Data Structure:

        ObjectInfo {
            startAddress;
            size;
            callSite;
        }

        BasicPageAccessInfo {
            firstTouchThreadId;
            isPageContainMultipleObjects;  // 1bit
            accessNumberByFirstTouchThread;
            accessNumberByOtherThread;
            // if isPageContainMultipleObject
            //       && accessNumberByFirstTouchThread > a threshold
            //       && accessNumberByOtherThread >> accessNumberByFirstTouchThread
            //       go to CacheLineDetailedInfoForPageSharing

            partialOccupiedCacheLineBitMask;
            cacheLineWritingNumber[CACHE_LINE_NUMBER];
        }

        CacheLineDetailedInfoForPageSharing {
            totalAccessThreadBitMask;
            accessNumberByFirstTouchThread;
            accessNumberByOtherThread;
        }

        CacheLineDetailedAccessInfo {
            totalInvalidationNumber;
            totalAccessThreadBitMask;
            accessThreadBitMaskAfterWriting;
            wordDetailedAccessInfo;  // only touch this if this cache line is partially occupied.
        }

        wordDetailedAccessInfo {
            // one thread or more thread ? one byte for each word.
            isTouchedBySingleThread[WORDS_NUMBER];
        }

        CacheLineDiagnosedInfo {
            cacheLineStartAddress;  // fix-helping info
            isFalseSharing;  // if the parts belong to this object is only accessed by single thread
            totalAccessThreadBitMaskForCacheLine;   // fix-helping info
            totalInvalidationNumberForCacheLine;  // how serious this is
        }

// page level: how to determine cause by allocator or application
        PageDiagnosedInfo {
            pageStartAddress;
            accessThreadBitMaskInPage;
            accessThreadBitMaskBySelf; // fix-helping info

            totalAccessNumberInFirstTouchThread;
            totalAccessNumberInOtherThread;
            // we can say it is caused by allocator if :
            // totalAccessNumberInFirstTouchThread >> accessNumberBySelfInFirstTouchThread
            // which means the firstTouchThread is high possibly controlled by other objects.
            accessNumberBySelfInFirstTouchThread;
            accessNumberBySelfInOtherThread
        }

        ObjectDiagnosedInfo {
            ObjectInfo;

            List<CacheLineDiagnosedInfo> seriousCacheLineInfo;

            List<PageDiagnosedInfo> seriousPageInfo;
        }


Global Variable:

        HashMap<ObjectInfo> objectInfoMapping; (because they are only accessed in the malloc and free function)

        HashMap<> diagnosedInfoMapping; (key:callSite, only accessed in the free function)

        ShadowMapping<BasicPageAccessInfo> basicAccessMapping;

        ShadowMapping<CacheLineDetailedInfoForPageSharing> cacheLineDetailedInfoForPageSharingMapping;

        ShadowMapping<CacheLineDetailedAccessInfo> cacheLineDetailedAccessInfo:(shadow mapping: mmap a big chunk of memory form the first beginning)


Processing flow:

        malloc(){
            1. insert ObjectInfo into objectInfoMapping hashMap.   (very low overeheads)
            2. set ifTotalCoveredBySingleObject in PageAccessInfo;
            3. return the allocated address.
        }

        handleAccess(){
            1. increase the write number in the corresponding cache line slot.  (very low overeheads)
            2. if the write number > certain threshold                          (very low overeheads)
                      a. if this cache line is marked as partially occupied cache line:
                            allocate wordDetailedAccessInfo for it.
                      b. if writing access:                  (may cause 2-3 cache line miss)
                            1. accessThreadBitMaskAfterWriting = 1 << lastWriteThreadId (which means invalidate cache lines for other threads)
                            2. update totalCopyNumber in the cache level;
                            3. update totalAccessThreadBitMask in cache level
                            4. if marked as partially occupied cache line
                                    update totalAccessThreadBitMask in word level.
                      c. if reading access:                   (may cause 2-3 cache line miss)
                            1. if followingAccessThreadBitMask & (1 << threadId) ==0: (which means first time access for this thread)
                                    a. update accessThreadBitMaskAfterWriting
                                    b. if marked as partially occupied cache line
                                            update totalAccessThreadBitMask in word level.
        }

%        update totalCopyNumber() {
%            if currentThreadId == pageFirstTouchThreadId:
%                    copyNumber++;    (local access)
%            else:
%                    copyNumber+=1.5;  (remote access)
%        }
%        // why this is not a good idea for page? it can tell the problem, but can not help fix the problem.

        sigsegv handler() {
            update the first touch thread id and the first touch address;
        }

        // data structure for the final results. how to collect the page level info....
        free() {
            1. get ObjectInfo from the hashMap-objectInfoMapping. (very low overeheads)
            2. collect DiagnosedInfo data from the accessMapping. (can be easily summed up according to current data)
            //only if the object size is larger than cache line, collect the PageDiagnosedInfo.
            // because in this case, the prime problem is from cache level, not page level.
            3. store the diagnose results to the diagnosedInfoMapping.
            4. remove object from the hashMap-objectInfoMapping.
            5. free the object.
        }
